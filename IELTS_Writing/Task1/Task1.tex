\documentclass[conference]{IEEEtran}
\IEEEoverridecommandlockouts
% The preceding line is only needed to identify funding in the first footnote. If that is unneeded, please comment it out.
\usepackage{cite}
\usepackage{amsmath,amssymb,amsfonts}
\usepackage{algorithmic}
\usepackage{graphicx}
\usepackage{textcomp}
\usepackage{xcolor}


\begin{document}

\title{IELTS Writing Task 1\\}

\author{\IEEEauthorblockN{Liu Zhaohong}}

\maketitle

\section{Dynamic Graph}

\subsection{Line Graph(past time)}

\textbf{The graph below shows the quantities of goods transported in the UK between 1974 and 2002 by four different modes of transport.}

\textbf{Summarize the information by selecting and reporting the main features, and make comparisons where relevant.}

\begin{figure}[htbp]
    \centerline{\includegraphics[width=1.4\columnwidth]{images/Screenshot from 2022-12-03 15-25-38.png}}
\end{figure}

The graph compares the volumes of goods delivered by four means of transport in the UK during the period 1974 to 2002.

The amount of goods transported by road was the largest, with around 70 million tonnes in 1974. 
It sharply increased to about 100 million tonnes in 2002. 
The figure for water was lower throughout the period: 
it rose from around 40 million tonnes in 1974 to about 58 million tonnes in 1980 and then stood at this level until 1998, 
before climbing to over 65 million tonnes in 2002.

The amounts of goods transported by the other two means of transport were lower. 
There was a dramatic fall in the figure for rail transportation from around 40 million tonnes on 1974 to 30 million tonnes in 1984, 
although it increased to over 40 million tonnes in 2022. 
The pipeline for transporting goods saw a steady growth from approximately 5 million tonnes in 1974 to over 20 million tonnes in 1995, 
and then it remained at this level in the rest of the period. Despite the growth, it was the least popular means of transport.

Overall, almost every means of transport in UK saw an upward trend in the goods delivery, 
while there was a different pattern in rail. Road transportation delivered more goods than any other means of transport.

\begin{itemize}
    \item compare
    \item transport=deliver, good
    \item by 4 means of transport
    \item during the period 1974 to 2002
    \item amount, figure, amount for goods, figure for water
\end{itemize}

\subsection{Line Graph(past to future)}

\textbf{The graph below gives information from a 2008 report about consumption of energy in the USA since 1980 with projections until 2030.}

\textbf{Summarise the information by selecting and reporting the · main features and make comparisons where relevant.}

\begin{figure}[htbp]
    \centerline{\includegraphics[width=1.0\columnwidth]{images/Screenshot from 2022-12-04 14-19-49.png}}
\end{figure}

The line graph shows the use of different sources of energy in the US over a 50-year period between 1980 and 2030.

Petrol and oil are the most important energy sources from 1980 to 2030. 
The consumption of these two fuels increased steadily and is expected to grow to 50 quadrillion units in 2030. 
There will be a similar trend in the consumption of coal, rising from around 17 q in 1980 to 30 q in 2030. 
It will become the second most important fuel. 
The consumption of natural gas saw a slight increase to about 25 q units in 2015 and will remain at this level in the rest of the period.

In contrast, the consumption of new energy sources, including nuclear, solar/wind and hydropower, is much lower. 
There will be a steady rise to nearly 9 quadrillion in nuclear consumption in 2030.  
The consumption of solar/wind will climb to over 5 quadrillion, while the figure for hydropower is predicted to fall to around 3 quadrillion.

Overall, fossil fuels will still be more important than environment-friendly alternative in the US. 
The energy production for all resources is expected to rise to various degrees, whereas the use of hydropower will show a different pattern.

\subsection{Pie Chart}

\textbf{Surveys conducted in 1982 and 2002 show different pictures of what motivate students to choose a college or university in the UK}

\textbf{Summarise the information by selecting and reporting the main features and make comparisons where relevant}

\begin{figure}[htbp]
    \centerline{\includegraphics[width=1.0\columnwidth]{images/Screenshot from 2022-12-04 17-38-02.png}}
\end{figure}

\begin{figure}[htbp]
    \centerline{\includegraphics[width=1.0\columnwidth]{images/Screenshot from 2022-12-04 17-38-13.png}}
\end{figure}

The charts present the findings of a survey about what British students considered when choosing an university in two different years.

Pie charts compare the factors that students considered when choosing a university in the UK in two different years. 

suitable courses and degrees were the most popular consideration, although the proportion of students who cited this reason dropped from 40\% to 34\%. 
The percentage of students who considered costs was the second highest, but there was a sharp decrease to 5\%. 
Similar change happened to location factor either. The figure for those considering the distance to parental home declined from 19\% to 16\%.

In contrast, the proportion of students who paid attention to the reputation of the university was as low as 5\% in 1982, 
but this factor saw a significant rise to 25\%. 
There was a remarkable change in the percentage of people who valued amenities, social activities and sports facilities, rising from 5\% to 18\%. 
The figure for admission criteria remained basically unchanged(3\% in 1982 and 2\% in 2002).

Overall, more students considered suitable courses and degrees than those who focused on other factors. 
The percentages of students who considered fame of the university or amenities and sports facilities increased, 
while the figures for other factors showed opposing trends.

\subsection{Table}

\textbf{The table shows the amount of waste produced by different countries in 1980, 1990 and 2000}

\textbf{Summarise the information by selecting and reporting the main features, and make comparisons where relevant}

\begin{figure}[htbp]
    \centerline{\includegraphics[width=1.1\columnwidth]{images/4.png}}
\end{figure}

The table provides information about the waste created in six countries in three separate years.

The waste produced by US was the highest, and it increased dramatically to 4005 million tonnes in 2000.
The figure for Japan was much lower, and there was a slight rise to 52 million tonnes.
Although no figure was given for Korea in 1980, the amount of waste produced by this country was huge(31 and 19 millions).

In contrast, the amounts of waste produced by other countries were much lower.
The figure for Ireland was the lowest, and there was no information for the year 1990.
Poland saw a slight increase from 2 million in 1980 to 5 million in 2000.
The figure for Portugal remained basically unchanged, climbing to 10.1 million and then decreasing to 9.7 million.

Overall, the US created more waste than any other country.
All countries created more waste during this period while Korea and Portugal saw different trends.

\subsection{Pie charts(Two objects to describe)}

\textbf{The pie charts below show units of electricity production by fuel source in Australia and France in 1980 and 2000.}

\textbf{Summarise the information by selecting and reporting the main features and make comparisons where relevant.}

\begin{figure}[htbp]
    \centerline{\includegraphics[width=1.1\columnwidth]{images/Screenshot from 2022-12-04 21-53-13.png}}
\end{figure}

The charts compare two countries in terms of the electricity produced by different fuels in 1980 and 2000.

The amount of electricity generated by coal was highest in Australia, and it increased from 50 to 130 units in 2000.
The figure for coal in France was much lower(25 units in both years).
There was a significant growth in the electricity created by nuclear power in France(from 15 units to 126 units).
The figure for hydro power increased from 20 to 36 units, although it was rather low in France(only 2 units in 2000).

Australia saw a drop in the electricity created by oil to 2 units in 2000, while France saw a rise to 25 units.
The figures for natural gas showed the same trend in these two countries, drooping to 2 units.

Overall, Australia relied more on coal than on any other energy source.
The amount of electricity generated by these two fuels increased in respective countries.

\subsection{Pie charts(Two objects to describe)}
\textbf{The charts below give information on the ages of the populations of Yemen and Italy
in 2000 and projections for 2050}

\textbf{Summarise the information by selecting and reporting the main features and make
comparisons where relevant.}

\begin{figure}[htbp]
    \centerline{\includegraphics[width=1.1\columnwidth]{images/Screenshot from 2022-12-04 22-22-15.png}}
\end{figure}

The pie charts show the proportions of 3 different age groups in two countries in 2000, 
as well as projected figures for the year 2050.

In Yemen, people aged 15 to 59 are projected to represent the largest proportion of the population at 57.3\% in 2050,
up from 46.3\% in 2000.
A totally different pattern is expected to be seen in Italy for this age group,
with the figure dropping significantly from 61.6\% to 46.2\%,
although in both years, this age group was and will be the biggest one.

Both countries are likely to see an increase in the population aged 60 or older.
In Italy, the projection is that over-60s make up 42.3\% of the entire population, 
significantly higher than 24.1\% in 2000.
In Yemen, the growth is much smaller-from 3.6\% to 5.7\%.

As for the youngest group, these two countries may experience the same change.
The proportion of those aged 14 or younger in Yemen is likely to decline remarkably from 50.1\% to 37\%,
while only a modest decrease is predicted for Italy from 14.3\% to 11.5\%.

Overall, both countries are projected to see their population grow older in 2050, 
with a smaller proportion of those under the age of 14 and a larger proportion of those aged 60 or above.
In spite of these changes, the projection is that the 15-59 age group will be the biggest one in both countries.

\subsection{Bar Chart(Two objects to describe)}

\textbf{The charts give information about the proportions of boys and girls of a school who
achieved high grades (A or B +) in respective courses.}

\textbf{Summarise the information by selecting and reporting the main features, and make
comparisons where relevant.}

\begin{figure}[htbp]
    \centerline{\includegraphics[width=1.1\columnwidth]{images/Screenshot from 2022-12-04 22-40-33.png}}
\end{figure}

The bar chart show the changes in the percentages of boys and girls gaining impressive grades in different subjects in 1996 and 2000.

Humanities saw the biggest increase in the proportion of high-achieving boys, with the figure nearly doubling to 43\% or so.
By comparison, the figure for girls drooped by 8\% to 25\%.
Similar changes happened to arts, in which the percentage of boys who gained high scores increased significantly to 21\% and
the figure for girls decreased slightly to 25\%.
Languages were another course which experienced a drop in the percentage of high-achieving girls, down to 31\%,
and the proportion of high-achievers for boys remained unchanged at about 21\% in this course.

There were similar patterns in the performance in science and maths for both grades.
Around one third of boys gained satisfying grades in these science courses in 1996-
higher than the figures for other subjects-but only around 18\% and 17\% reached this level in 2000.
In contrast, the proportion of high-achieving girls in these two courses increased to 11\% and 15\% respectively,
although these figures were still far lower than those for the other three subjects.

Overall, there were clear gender differences in all these five subjects: 
science subjects saw an improvement in the performance of girls, while the arts and humanities had a higher of proportion of high-performing boys.
In 2000, boys outperformed girls in every subject except the arts.


\section{Static Graph}

Compared with dynamic graphs, static graphs have no trend to conclude, only higher or lower.

some sentences in comparison:
\begin{itemize}
    \item Country A has a bigger population than country B
    \item The population of A is bigger than that of country B
    \item Compared with A, B has a small population(in contrast to)
    \item The population of A is different from that of B(similar to, the same as)
\end{itemize}

express times:
\begin{itemize}
    \item twice, three times \dots
    \item The number of A is three times as high as that of B
\end{itemize}

\begin{figure}[htbp]
    \centerline{\includegraphics[width=1.1\columnwidth]{images/Screenshot from 2022-12-05 11-00-39.png}}
\end{figure}

\begin{figure}[htbp]
    \centerline{\includegraphics[width=1.1\columnwidth]{images/Screenshot from 2022-12-05 11-00-46.png}}
\end{figure}

to express age:
\begin{itemize}
    \item people aged 30-39
    \item people aged between 30 and 39
    \item people in their thirties
    \item 30-39 age group
    \item 30-39-year-old people
\end{itemize}

\subsection{Bar chart}

\textbf{The chart below contains information provided by Australia tertiary institutions about the
percentage of male and female students who enrolled in different subjects in 1995.}

\textbf{Summarise the information by selecting and reporting the main features and make
comparisons where relevant.}

\begin{figure}[htbp]
    \centerline{\includegraphics[width=1.0\columnwidth]{images/Screenshot from 2022-12-05 11-04-45.png}}
\end{figure}

The bar chart shows the proportions of men and women who studied different university subjects in the year 1995 in Australia.

Engineering saw the biggest gender gap, with nearly 19\% of men choosing it,
(Engineering attracted the highest proportion of male students)
while the percentage o females who studied this subject was the lowest(less than 3\%).
Computer was also much more popular with men than women(10\% and 4\% respectively).
Similarly, around 13\% of men studied Math and Sciences, around twice as high as the figure for women.

Health, by comparison, attracted more women than men (15\% in comparison with less than 4\%).
A similar pattern was seen in Education, chosen by 10\% of women and around 6\% of men.
Arts and Humanities were the favorite subject for females, with up to 19\% of them choosing this subject, compared with 13\% of men.

In contrast, the gender gap was narrow in two business subjects.
13\% of women and 11\% of men enrolled in Accounting and Economics, 
and smaller proportions of men and women selected Business Studies(6\% and 8\% respectively).

Overall, men were more likely to choose Engineering or Science subjects than women, 
while the Arts and Education were favored by a larger proportion of female students.
In Business subjects, the gender difference was less clear.

\subsection{Bar chart(2 graphs)}

\textbf{The charts below show the main reasons for
study among students of different age groups and the amount of support they received from employers.}

\textbf{Summarise the information by selecting and reporting the main features and make
comparisons where relevant.}

\begin{figure}[htbp]
    \centerline{\includegraphics[width=1.0\columnwidth]{images/Screenshot from 2022-12-05 11-36-14.png}}
\end{figure}

The bar charts provide information about the proportion of students who studied for career reasons or interest reasons and 
also the percentage of students who received support in time and economy from employers for education.

The vast majority of people under 26 studied for career advancement, 
compared with only 10\% motivated by personal interests.
The proportion of students who gave priority to career, however, 
dropped with age and reached 40\% for students in their 40s, equal to the proportion of people of the same age studying for personal interests.
In the oldest age group(those over the age of 49), up to 70\% of people were driven by personal interests,
more than three times higher than those by job prospects.

Employers had a strong interest in investing in young people, 
with more than 60\% of those under 26 and 50\% of 26-29-year-old students receiving assistance from their bosses.
The level of employer support was the lowest for the 30-39-year-olds, at 33\% only, 
compared with higher higher figures for the two older age groups (about 35\% of the 40-49 age group and 42\% of those aged 49 or over).

Overall, younger age groups(those under the age of 39) were more likely than their older counterparts to study for career-related reasons.
The likelihood of gaining assistance from employers was also relatively high before students turned 30.

\section{Flow Chart / Process diagram}

\subsection{Flow chart}

\textbf{The diagrams below show the stages and equipment used in the cement-making process,
and how cement is used to produce concrete for building purposes.}

\textbf{Summarise the information by selecting and reporting the main feature s and make
comparisons where relevant.}

\begin{figure}[htbp]
    \centerline{\includegraphics[width=1.0\columnwidth]{images/Screenshot from 2022-12-05 12-18-38.png}}
\end{figure}

The charts illustrate how cement is produced and how it is used for the production of concrete.

The first step of cement production involves the crushing of limestone and clay in a crusher to create powder,
which is subsequently mixed in a mixer(cylinder).
The mixture then goes through a rotating heater, which is heated with a device at the bottom, and then falls onto a conveyor belt.
At the end of the belt, the mixture is ground by a grinder(heavy roller) to create cement, which is packaged in bags.

As shown in the second chart, cement represents 15\% of the raw material of concrete production, 
while water, sand and small stones make up 10\%, 25\% and 50\% respectively.
All there materials are blended in a concrete mixer, which rotates in a clockwise direction until concrete is made.
There is possibly just one step-mix the assembled raw materials.

To Summarise, cement production consists of five steps, and its final product is combined with three other ingredients to manufacture concrete.
The former process is much more complex, as it involves the use of various devices and facilities.

\section{Map Graph}

\subsection{Big map}

\textbf{The maps below show the center of a small town called Islip as it is now,
and plans for its development.}

\textbf{Summarise the information by selecting and reporting the main features, and make comparisons
where relevant.}

\begin{figure}[htbp]
    \centerline{\includegraphics[width=1.0\columnwidth]{images/Screenshot from 2022-12-05 12-40-59.png}}
\end{figure}

The diagrams illustrate the proposed changes to the central area of a town named Islip.

One major development, according to the plan, is tha pedestrianisation of the main road which currently runs through the center from west to east.
Another principal change is the building of a dual carriageway around the center,
which can divert traffic and make the central area pedestrian-only.

There is a row of shops along either side of the current main road. 
In the future, while the shops along the south side remain, those along the opposite side are set
to be demolished to make way for a bus station, shopping centre, car park and new housing area. 
To the south of the shops is a park, which is expected to be reduced in size to allow for new housing developments. 
Finally, the school and housing on the southern edge of the centre are likely to remain, but outside the carriageway.

Overall, the major developments proposed to the Islip town center include the construction of a new ring road and the pedestrianisation of the main road, 
which will lead to subsequent changes to transportation, shopping facilities and residential areas.

\end{document}