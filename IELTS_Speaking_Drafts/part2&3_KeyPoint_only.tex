\documentclass[conference]{IEEEtran}
\IEEEoverridecommandlockouts
% The preceding line is only needed to identify funding in the first footnote. If that is unneeded, please comment it out.
\usepackage{cite}
\usepackage{amsmath,amssymb,amsfonts}
\usepackage{algorithmic}
\usepackage{graphicx}
\usepackage{textcomp}
\usepackage{xcolor}

\begin{document}

\title{IELTS Speaking Part 2\&3\\
{\footnotesize \textsuperscript{*}Note: 9-12 season query}}

\author{\IEEEauthorblockN{Liu Zhaohong}}

\maketitle

\begin{abstract}
speaking draft for IELTS test part 2 and 3
\end{abstract}

\section{People}
\subsection{Describe a person you know who is from a different culture}
Leo Messi;
    \begin{itemize}
        \item Who he/she is
        \item Where he/she is from
        \item How you knew him/her
        \item How you feel about this person
    \end{itemize}

For me that person should be Leo Messi. I think almost everybody knows him or ever heard about him.
He is a football player from Argentina, with the nationality of both Spain and Argentina.
I heard stories about him in my primary school, probably the year 2010.
At that time Messi was still young, but he was already famous in the football area;

Firstly I thought Messi was a selfish man who only dribble the ball by himself and never passes it to his teammates,
and know nothing about the teamwork. But now, actually since 2014, I have become a big fun of him.
2014 is the year when the Brazil world cup were held.
If you are familiar with that cup, you must remember that the final was between Germany and Argentina,
and unfortunately, Argentina lost that game with Mario Gotze scored at 114 minute.
It's really sad because Messi and his teammates had several good chances before that goal.
However, they didn't make it, I mean the chance didn't convert to a score.
Sadly that match was just a start, Argentina lost another American cup final in the last 2 years.
For Messi himself, Barcelona was turned over twice at the European Champions League.

I remember once during that time, Messi quitted the nation team because suffering from so many failures.
But finally he came back again and led Argentina team for the 2018 World cup, sadly lost again.

Looking at Messi's career, you will see with such talents and diligence of him, failure still could
be a common thing. I like his dribbling skills, which is really attractive, and I also like his humble personality.
But most of all, I like his behavior that a man can be defeated, but sooner or later he should stand up
then facing and overcoming those obstacles.
This give me the bravery to handle my daily failure, too.

\textit{1. How can we get to know people from different cultures better?}

first of all, respect; which means keep a humble profile and better know some of his or her culture.
Especially the taboo in that culture;

Second one is to express any idea directly; That is to be simple, meet anything confusing, be free to say that
I don't know then ask;

For more or detailed understanding, read books or watch news. Movies is also a good way.

\textit{2. What are the advantages and disadvantages of cultural diversity?}

Advantages include that, one can know different kinds of culture and that could be interesting;
different culture brings various custom, life style, these are the treasure of all human being.

Disadvantage might be, more possible conflict.
But that is not for certain. Different culture might take misunderstanding, and even account for fight and blood.
However, that can be avoided by peaceful communication.   

\textit{3. How can traditional culture and other cultures coexist?}

by finding similarity and ignore difference as much as possible.
focus on the similarity or the double-win staff.


\subsection{Describe a person who is fashionable}
My girlfriend;

\begin{itemize}
    \item Who he/she is
    \item What he/she does
    \item What kind of clothes he/she wears
    \item And explain why you think this person is fashionable
\end{itemize}

Okay I'm gonna talk about my girlfriend, the closest fashionable person to me.
currently she is a postgraduate student like me, and we were schoolmates during the undergraduate period.
There is a slang in China that a swift horse needs a high quality saddle to utilize its strength, so does a man, 
who needs suitable clothes.

I think a woman, or paticularly my girlfriend, has 2 accounts;
one account is used in campus, another one is for entertainment, and they are quite different.
In campus she usually wears simple and comfortable clother, like sports clothes.
I mean a pair of sneakers, a pair of loose pants, some daily T-shirt or hoodie.
However, that's another story for the recreation part;
That include dress, all kinds of coats, jackets, shoes and so on.
By the way, also including different colors of dyed hair;

While the most fashionable thing about her is not the clothes that she wear, but the combination of them.
I mean, maybe each clothes is a simple one, without complex design or various logos and icons, some of
them even seems plain and simple.
But my girlfriend got the magic to organize them (mixture and fusion and integration).
It's like she is solving a math problem that by which means of arranging the combination will bring highest fashion score.

So fashion is kind of knowledge and witness I guess.
Unfortunately I'm the one who know nothing about fashion, so I'm lucky to have her as my coach on dressing.

\textit{1. Are older people as fashionable as young people?}

I think there is no coorelation between age and fashion.
My opinion is they may have diferent definition about what fashion is.
For example, people born in 1950s used to wear the same green uniform as the military,
to express their respect to PLA and that's their fashion.
China keeps developing, after the reform and opening up we accecpted more western cultures;
Till now, we define our own fashion, maybe each one have their own fashion.

\textit{2. Are women more fashionable than men?}

It seems so but I think this is a problem that does't have that obvious link with gender.

We may all believe that women spend more time on make up, to preen themselves.
But actually some men do it either.
It's not about gender about only related to whether this persion is a fashionable one or not.

\textit{3. Why is fashion more important to some people?}

I think being fashion make them feel more confident, and being confident means one can express him or herself
more bravely.

For example, being fashion for some introvert people is kind of mental hint, telling them they are good and be
free to say anything or do something, I guess.


\subsection{Describe a person you know who loves to grow plants}
some source said plants=vegetables and fruits, some add flowers

My dad;

\begin{itemize}
    \item Who this person is
    \item What he/she grows
    \item Where he/she grows those vegetable/fruits
    \item And explain how you feel about this person
\end{itemize}

That person is my dad. He was born in 1970s, and grew up in a small village in northern China.
His parents, namely my grandmother and father were farmers.

So he did a lot of farming works when he was young, and he kept studying during that time, 
finally with the reform and opening up and his hard work, he entered a agricultural university, majoring in vest.
But he didn't end up to be a vest, he was distributed to an anti-virus department after graduation.
Now he is working in a normal government department, nothing related to planting at all.

However, during these years, he still keeps his planting skills.
We lived in the top floor of a flat in the past. And we had an outdoor balcony there, 
so my father used that place as a natural field for planting Chinese chives(vegetables) and grapes.
and I remember each summer when I was young, we can eat our own grapes, although so many of them were eaten by
passing-by birds and bees, so actually only a few left for us.
We once kept a dog, my dad found him near a dustbin homelessly so he took him back home and raised him.
and that dog destroyed many of his vegetables, dad didn't blame him, we just didn't have our own chives to make
dumplings those days.

Except for the balcony stuff, dad also have many flowers inside the room, each time they blossom dad will take
pictures and share that to his friends.

I think the purpose of planting has changed for him.
In the past when he was young, he had to do farm work to help his family.
When he grew older and got to work in the city, he had no need for that farm work,
till now, watering the flowers and other works look like a hobby, or an addiction for him.
I guess that's because there is farmer's blood in his vein.

\textit{1. What do you think of the job of being a farmer?}

I think it's quite normal, being a farmer is also a job in the whole social life.
And I believe that there shouldn't be discrimination against farmers, ranking them as the lowest of this society;
reason is simple cause no farmers, no food, no anything.

\textit{2. Are there many people growing their own vegetables now?}

Farmers still occupy a considerable part of population in China, and they surely grow their vegetables.
But for people living in cities, I don't think many of them have the time, skill and space to grow their own vegetables.


\subsection{Describe your favorite childhood friend}
My neighbor;

\begin{itemize}
    \item Who he/she is
    \item Where you met each other
    \item What you often did together
    \item And explain what made you like him/her
\end{itemize}
% who and where met
The childhood friend that I'm gonna talk about is my neighbor's son. He is also my neighbor either.
You know this neighbor I'm saying is not we lived in the same building but, in the same floor and door to door.
He is 4 years older than me, and I don't remember when we met.
Because I guess when we met is really a historical problem, maybe his mother took him to see my mother and me
just after I was born safely.

% often did together
We did a lot of things together, but there are some remarkable ones.
When I was 7 or 8, in my primary school period. Some buildings were dismantled near my home, because
they are too old and the city wanted to build some new stuff.
You know, the dismantle and construct departments are 2 different department.
So there is quite a time span that the ruin stayed there. And one night, we picked up some branches
, papers and a lighter to set up fire for barbecue.
and the food we baked is a corn.
I remember the lighting process is quite hard cause it's a little windy so we managed to light the paper first;
then using the burning paper to light branches, finally using steady fire to bake corn.
Actually that corn wasn't that delicious, but the process is interesting for me.

% digress a little
We also used to make paper plane then throw them at my outdoor balcony, to see whose plane could fly further.
and sometimes rode bicycle around the city in some nights. 

% what made like
To be honest, set up a fire on a mess of ruins is really dangerous. And except for that frenzy stuff,
he is also kind of my path-leader, after all he is older than me, so he did teach me some experience
that I used later.

\textit{1. Why do people lose contact with their friends after graduation?}

It has a lot of reasons;
First, we may spend more time with people around us. so without particular attention,
we might just forget to communication with old friends.

Secondly, if people go to work after graduation, the work load may cut greatly on their time
for social, causing the losing contact of old friends.

\textit{2. How does modern technology influence friendship?}

for truly friendship, technology makes 2 people get to know each other more quickly and precisely.
It's beneficial for keeping the friendship.

In the meantime, the time or cost to interrupt someone is also declined. If you make a call to your
friend directly while he is having an important conference with his volume on, that could be a miserable.

\textit{3. Do you think people's relationship with friends will change when they get older?}



\subsection{Describe a person who inspired you to do something interesting}

\begin{itemize}
    \item Who he/she is
    \item How you knew him/her
    \item What interesting thing you did
    \item And explain how he/she inspired you to do something interesting
\end{itemize}



\section{Object}
\subsection{Describe a photo you took and you are proud of}

\begin{itemize}
    \item \textbf{When you took it}\\
    When I was a senior student in my undergraduate period, 2021's winter, 
    third year in my undergraduate life.

    \item \textbf{Where you took it}\\
    An old temple in Xi'an city, the foot of Qinling Mountains.

    \item \textbf{What is in the photo}\\
    My girlfriend stood under the roof of a corridor in that temple.

    And I was standing in the heavy snow, taking a picture of her.
    
    We were missing, nobody else was there, at that time she was looking for where should we go then.
    so she just lifted her head slightly and looked at far side;

    That is a traditional temple of Chinese Buddism, there were pine trees behind the corridor, 
    they were all covered with sonw;
    
    snow kept falling down, and the second when I took the shot, a piece of sonw just passed by her eye. 
    Also you can see multiple pieces of snow in that picture. 
    
    \item \textbf{Explain why you are proud of it}\\
    in Chinese poems, we judge whether a scene is beautiful not only by the scenery, but also by the further emotion behind it;
    
    I think that picture express kind of emotion that although there is difficulty, namely the heavy snow and cold weather;
    and no one else was there to offer you a hand.
    then we have to think and look positively; although the weather is bad, it also gave us a beautiful picture.
\end{itemize}

\textit{1. Why do some people like to record important things by photos?}

\begin{itemize}
    \item  digital photos can be saved forever
    \item  convenient and easy to photo a schedule rather than writing down details
\end{itemize}

\textit{2. What can people learn from historical photographs?}

\begin{itemize}
    \item the truth of history, rape of Nanking
    \item clues to some unknown things
\end{itemize}

\textit{3. Is taking photo the best way to remember something?}

It depends.

\begin{itemize}
    \item knowledge, s simple photo might be useless
    \item cultural relic, function can't be replaced by photo
    \item a journey. photo is a good way.
\end{itemize}

\subsection{Describe an object that you think is beautiful}
describe a beautiful object
% 就写微博看到的火烧日照金山
\begin{itemize}
    \item \textbf{What it is?}\\
    It is a video about a peek of snow mountain.

    \item \textbf{Where you saw it?}\\
    I saw on the internet.

    \item \textbf{What it looks like?}\\
    Not sure about the time, morning or dusk.
    the  mountain is surrounded by rosy clouds, actually it's between golden and red.
    windy, so clouds keep moving.
    and it looks like the peek is set up fire and burning.

    \item \textbf{And explain why you think it is beautiful}\\
    contrast is attractive.

    it contains both static and motion, austere and feisty.

    the snow mountain is a static object, while clouds is in movement;
    the snow mountain consists of black and white, austere;
    while the clouds is red and golden, they are feisty.

    the contrast makes the video so beautiful.
\end{itemize}

\textit{1. Do you think there are more beautiful things now than in the past? Why?}

yes, cause we have more definition of beautiful things now.
\begin{itemize}
    \item in the past, they have relatively strict definition about what is beautiful
    \item and the existence of emperor, his will and preference could be so called beautiful
    \item currently, globalization makes one can get access to other culture
    \item the definition of beauty now is various
\end{itemize}

\textit{2. What beautiful scenic spot are there in your country?}
that will be plenty
\begin{itemize}
    \item northern China, pine forests and snow covered village
    \item southern China, sea, beach
    \item southwest China, rain forests, snow Mountains
\end{itemize}

\textit{3. Where do you think people usually come into contact with beautiful things?}
\begin{itemize}
    \item internet, Apps on smart phone
    \item friends, chat groups with similar interest
    \item TV, radio and so on
\end{itemize}

\textit{4. Why do you think people create beautiful things?}
\begin{itemize}
    \item work(painter, artist), aim is to create beauty
    \item hobby, occasionally create beauty, love doing with skill polishing, come up with
    \item total occasion, chancy
\end{itemize}

\subsection{Describe a movie you watched recently and would like to watch again}
% 万里归途吧就
\begin{itemize}
    \item \textbf{What it was about}\\
    It's a story about some diplomatic officers, evacuating oversea Chinese and took them back home.
    \item \textbf{Where you watched it}\\
    Two months ago, during the nation holiday, at my girlfriend's city.
    \item \textbf{Why you like it}\\
    Firstly, the main actor Zhang is a good role model in my opinion; (work hard, diligent, the eating noddle scene)
    
    Secondly, the scene is real, I can feel the chaos and danger caused by war at that country.

    Finally, the story itself is based on a true story. Reflect on the difficulty our diplomatic officers had met and
    their skills and faith to bring Chinese citizens back home.
    \item \textbf{And explain why you would like to watch it again}\\
    First, this movie is really attractive and moving.

    Second, I'm near sighted and didn't take my glass with me. 1080p ended up with 480p.
\end{itemize}

\textit{1. What are the differences between movies at home and in the cinema?}
\begin{itemize}
    \item atmosphere;people laugh, cry, sigh, scream
    \item dark, more concentrated and emotional
    \item new movie only accessible for cinema
    \item home is more convenient, talk, eat, everything you want
\end{itemize}

\textit{2. Do you think actors(or actresses) are important to the success of a movie?}
\begin{itemize}
    \item scripts, plot, story. fundamental of a movie
    \item actors' display is decisive
    \item a good actor or actress can't save plots or bad design(thoughtful)
\end{itemize}

\textit{3. Why are there fewer people going to the cinema to watch movies?}
\begin{itemize}
    \item covid-19, dangerous
    \item life and work is busy
    \item high expenditure
\end{itemize}

\subsection{Describe a program you like to watch}
\begin{itemize}
    \item \textbf{What it is?}\\
    I recently finished a really fascinating tv series called seal team.
    \item \textbf{What it is about?}\\
    And it's about a seal team six, DEVGRU bravo team, about each team guy's operation, life, even death.
    \item \textbf{Who you watch it with?}\\
    Actually I watched it myself cause on one has the same interest like me.
    \item \textbf{And explain why you like to watch it?}\\
    As a military program, the most important thing about seal team is its plot.
    The show doesn't pay all attention to illustrate the fighting scene, instead, it spend half of time on
    model main roles, from their families and minds.
    For example I think the most successful role is bravo six Clay Spenser.
    He grew from a green team guy to a tier 1 operator with his own duty, but sadly he was shot accidentally
    by a police, I really love and hate the life of him.
\end{itemize}

\textit{1. What programs do people like to watch in your country?}
\begin{itemize}
    \item TV series; 7 pm, after dinner whole family on the television
    \item clips; short and funny mostly, on bus, subway or anywhere; tiktok
\end{itemize}

\textit{2. Do people in your country like to watch foreign TV programs?}
\begin{itemize}
    \item some young people might prefer foreign TV programs
    \item Like Netflix, HBO, some Japan and South Korea programs
\end{itemize}

\textit{3. Do students watch programs in class in China?}
\begin{itemize}
    \item seldomly or never
    \item the study programs can be taught by teachers, so unnecessary
    \item something about culture and history might be used
    \item or news
    \item never for those recreation shows
\end{itemize}


\subsection{Describe an important thing you learned(not at school or university)}
not given up, Messi example
\begin{itemize}
    \item \textbf{What it is?}\\
    the valuable thing I gained is from football match, and that's trust on hope and never give up
    \item \textbf{When you learned it?}\\
    I got to know this probably from a particular match, that was Barcelona against Paris Saint German.
    \item \textbf{How you learned it?}\\
    Barcelona tried their best at that match, but when the match only had 5 minutes left, they still had 3 goals behind.
    Can you imagine 3 goals in the last 5 minutes? 
    I would never believe this could happen until I saw it myself. 3 goals in 5 minutes and turned over the game.
    After that I always believes that anything could happen if you stick to it and also with some luck. 
    \item \textbf{And explain why is was important}\\
    Always be hopeful is a valuable thing because, for example, it gives you faith to trust that underdogs could come through
    in a football match, you won't just turn off the TV and go to bed.

    More meaningfully, it is also a good quality to never give up in any obstacles met in daily life.
\end{itemize}

\textit{1. What can children learn from parents? What about grandparents?}
\begin{itemize}
    \item life experience, accumulated in their lives
    \item how to treat people, deal with people
    \item grandparents might mollycoddle children
\end{itemize}

\textit{2. Do you think some children are well-behaved because they are influenced by their parents?}
\begin{itemize}
    \item parents can have some influence on children 
    \item mainly depend on themselves
    \item instruct, information, tell how to do, but whether do or not is a self decision
\end{itemize}

\textit{3. Is it necessary for adults to learn new things?}
\begin{itemize}
    \item necessary for each person
    \item knowledge is updating fast, no learn, be left behind
\end{itemize}

\subsection{Describe a story or a novel that you have read and you found interesting}
I really like the chapter A Study in Scarlet in Sherlock Holmes
\begin{itemize}
    \item \textbf{When you read it?}\\
    It was several years ago, maybe in my junior high school time.
    At that Time I was crazy about detective novels.
    \item \textbf{What the story or novel was about?}\\
    The story is about a man revenge for his wife, not exactly is wife I remember.(loved one)
    He got a kill list, each person on that list is responsible for his wife's death.
    And each time he erased a name from that list, he will leave a message by blood, saying revenge.
    But it's german I remember, and at last Sherlock got him, but sadly the criminal died a few days
    after he was captured, because these years he had got severe illness.
    \item \textbf{Who wrote it?}\\
    By Arthur Conan Doyle
    \item \textbf{And explain why you found it interesting}\\
    It is so attractive because it's quite ambivalent.
    The criminal, actually is a brave and good man I think, he had no choices and he just want
    those who hurt his loved one to pay.
    They escaped, and from the law, and after I knew how he struggled to kill those bustards, but sadly died after
    his revenge. I don't know my feeling exactly, it's so complex at that time.
\end{itemize}

\textit{1. How does technology help people tell stories?}
\begin{itemize}
    \item e-book, read out the story while you lying on the bed
\end{itemize}

\textit{2. Why are mystery novels so popular these days?}
\begin{itemize}
    \item because people are always looking for something unknown
    \item and mystery novels satisfy people's need of solving problems
    \item maybe peep secrets, eager to get answer
\end{itemize}

\textit{3. What kinds of stories do children like?}
\begin{itemize}
    \item past, fairy tale
    \item now, all kinds, cultural integration and globalization
\end{itemize}


\section{Place}
\subsection{Describe a popular place for doing sports(e.g.stadium)}
playground.
\begin{itemize}
    \item \textbf{Where it is?}\\
    There are some small football pitches that are quite suitable for sports not only include football.
    They are in my university.
    \item \textbf{When you went there?}\\
    usually in the night, after 7 o'clock, it quiet;
    daytime is because I have gym class;
    \item \textbf{What you did there?}\\
    have football class, for 3 semesters;
    joining football match at night;
    with my girlfriend after class each night we would take a walk across the whole campus, then sit for a while at the playground.
    have dinner there(buy snacks, sit, eat); sometimes there are people singing.
    \item \textbf{And explain how you feel about this place}\\
    place was wonderful and memorable;
    remind me of my undergraduate life.   
\end{itemize}

\textit{1. What are the benefits of children doing sports?}
\begin{itemize}
    \item benefit health, physical strength
    \item teamwork and communication skills
\end{itemize}

\textit{2. Do young people like to do sports?}
\begin{itemize}
    \item It all depends
    \item some have physical requirement on themselves, so they do sports usually
    \item some are too busy to exercise, mostly
\end{itemize}

\textit{3. Is it necessary to build sports venues?}
\begin{itemize}
    \item yes, infrastructures are essential for some certain sports
    \item China football bad performance link up with few football courts in each city
\end{itemize}

\subsection{Describe the home of someone you know well and you often visit}
\begin{itemize}
    \item \textbf{Whose home it is?}\\
    the home I often go but not my home, that can only be my grandparents' home.
    \item \textbf{How often do you go there?}\\
    I go there when I'm in vacation, maybe once a week.
    \item \textbf{What it is like?}\\
    It's not so far from my home, it's in the same city so maybe 30 minutes' drive.

    in a big village, kind of row houses, with only one floor and four buildings.
    like a square(shape)

    a quite big garden is surrounded by the four buildings. grapes, flowers and some vegetables are planted in the garden.

    a big pot, meal cooked enough for 5 people, use wood as fuel.

    raise a dog as housekeeper.
    \item \textbf{And explain how you feel about the home}\\
    a traditional Chinese village house;
    quite from houses in cities.

    relaxed.
\end{itemize}

\textit{1. What are the differences between houses or buildings in the city and in the countryside?}
\begin{itemize}
    \item cities: small, crowded, shelter from sunlight, convenient
    \item village: quite, far from bustle and hustle, relax, inconvenient
\end{itemize}

\textit{2. Do you prefer to live in the city or in the countryside?}
\begin{itemize}
    \item young-city: better job opportunities, more convenient, all kinds of facilities, noisy and busy
    \item old-village: quiet, relax, fishing or do some farm work
\end{itemize}

\textit{3. What are the safety risks in residential buildings in cities?}

most are safe
\begin{itemize}
    \item fire, gas seep
    \item electricity accident
\end{itemize}

\textit{4. Is it expensive to decorate a house or an apartment in the place where you live?}
\begin{itemize}
    \item won't happen to me
\end{itemize}

\subsection{Describe a place in your country that you would like to recommend to
travelers}
Xian old city
\begin{itemize}
    \item \textbf{Where is it?}\\
    It's in the center of Xian city, Northwest China.
    \item \textbf{What is it?}\\
    It's Xian old city, surrounded by city walls built in 500 years ago.
    \item \textbf{What people can do there?}\\
    climb to the top of city walls through stairs;

    some famous snack street; Xian feature food;

    different kinds of bars at night, drink cocktail of eastern and western integration flavor;
    
    ride bicycle on the top of city walls;
    \item \textbf{And explain why you would like to recommend it to travelers}\\
    Xian itself is a historical city, with 3 thousand years history.
    Many cultural relics.

    The food there can be found at somewhere else in China, but totally different; Once tasted, know which better.

    The best place to know about China, even better than Beijing.
\end{itemize}

\textit{1. Is it important to take photos while traveling?}
\begin{itemize}
    \item Yes, to have some visual memory after the journey.
    \item share with others and recommend that place to them
\end{itemize}

\textit{2. Can we trust other people's travel journals on the internet?}
\begin{itemize}
    \item like state estimation, pose and position
    \item so many wrong and bad data points
    \item to some extend, use more data
    \item so it means to check more journals and gathering information from all of them 
\end{itemize}

\textit{3. What factors affect how people feel about their journey?}
\begin{itemize}
    \item transport; clean, fast, comfortable or not;
    \item weather; terrible for a rainy day if you want to see sunrise on a mountain top;
\end{itemize}

\section{Event}
\subsection{Describe a problem you had while shopping online or in a store}
\subsection{Describe a time when you made a decision to wait for something}
\subsection{Describe a time when you received money as a gift}
\subsection{Describe a disagreement you had with someone}

\subsection{Describe an outdoor activity you did in a new place recently}
% recent can be long time ago
\begin{itemize}
    \item \textbf{What the activity is?}\\
    It was a football match that happened a week ago.
    \item \textbf{Who invited you to participate in it?}\\
    My roommate called me in, although I hadn't been playing football for 3 years. 
    \item \textbf{Whether you ask for help during the activity?}\\
    So it was hard for me;
    I used to be a forward,more precisely a shadow striker on the pitch.
    However, if the striker is not so proficient, he might lose the control of the ball easily;
    which means, your opponent would have a good chance to lead an attack, that is dangerous.
    So I had to ask him for several times to watch my six and help me.
    \item \textbf{And explain what change had in the activity}\\
    And of course, with his help, I was more confident in the game.
    He also told me to be focus on avoiding the physical confrontation, because I'm too thin.
    After that our side had became more aggressive, although it's not my goal, we won the game eventually.
\end{itemize}

\textit{1. What outdoor activities are popular in China?}
\begin{itemize}
    \item basketball, almost every big park have several basketball courts
    \item ping-pong, that's China's national ball, even small park can have a table for it.
    \item cycling, getting more and more popular due to traffic jam, tend to love bike
\end{itemize}

\textit{2. What are the differences between after-class activities done by young and older children?}
\begin{itemize}
    \item sadly they may now all have to keep studying, but there can be a little difference
    \item younger kids, have more opportunity to develop their interested skills or hobbies
    \item while elder children, teenagers, have to study all kinds of class to prepare for exams
\end{itemize}

\textit{3. Should young people try as many new activities as possible?}

It depends their thoughts.
\begin{itemize}
    \item try different activities can help you find what you like;
    \item but focusing on several certain activities can horn your skill and give you satisfaction through success
\end{itemize}

\subsection{Describe a time when you forgot an appointment}
\subsection{Describe a time when you shared something with others}
\subsection{Describe a time when you needed to search for some information}
\subsection{Describe a time when you saw a lot of plastic waste (e.g. in a park, on the beach
etc.)}
\subsection{Describe a time when you enjoyed an impressive English lesson}
\subsection{Describe a difficult thing you did and succeeded}




\end{document}
