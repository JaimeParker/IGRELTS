\documentclass[conference]{IEEEtran}
\IEEEoverridecommandlockouts
% The preceding line is only needed to identify funding in the first footnote. If that is unneeded, please comment it out.
\usepackage{cite}
\usepackage{amsmath,amssymb,amsfonts}
\usepackage{algorithmic}
\usepackage{graphicx}
\usepackage{textcomp}
\usepackage{xcolor}

% 添加中文支持包
%\usepackage[UTF8]{ctex}
%\usepackage{CJKutf8}

\def\BibTeX{{\rm B\kern-.05em{\sc i\kern-.025em b}\kern-.08em
    T\kern-.1667em\lower.7ex\hbox{E}\kern-.125emX}}


\begin{document}

\title{Speaking Part 1\\
{\footnotesize \textsuperscript{*}Note:
9-12 season query}
}

\author{\IEEEauthorblockN{Liu Zhaohong}
}

\maketitle

\begin{abstract}
speaking draft for IELTS test
\end{abstract}

\begin{IEEEkeywords}
IELTS, Speaking
\end{IEEEkeywords}


\section{Study/Work}

\subsection{Do you work or are you a student?}
Student + Degree + School + Major(engineering)

I'm a post graduate student at SJTU currently.

Majoring in aircraft control and information engineering. 


\subsection{What subject are you studying?}
Aircraft control and information engineering.

mostly it's about control stuff, like control law's designing and modelling of an aircraft

\subsection{Why did you choose that subject?}
I liked it in my high School and I was interested in planes, aircraft, something stuff

So, I chose this subject as my bachelor's Major and kept learning it in my master's study



\subsection{What would you like to do in the future?}
to be a robotics engineer



\subsection{What are the most popular subjects in China?}
computer science

seems with higher salaries than other subjects you can remember

\subsection{Do you think it's important to choose a subject you like?}
after the bachelor's time span, I'd say it depends on what life stage we are in

during bachelor time, one can choose subjects that are more universal.

For engineering students, they are math, physics, electronics and computer science etc.
the base for further study.

While in post graduate period, interest is rather important because
one need the interest to support his determination on doing research

\subsection{Are you looking forward to working?}
depends on the ??? of master degree

suitable for doing research or better for industry

\subsection{Do you like your subject? Why or Why not?}
no; compulsory subjects are not reasonable for us 

control vs. material mechanics

\subsection{Do you prefer to study in the mornings or in the afternoons?}
mornings

eat a lot at lunch, and digesting need more oxygen so I always feel sleepy

\subsection{Is your subject interest to you?}
mostly not, but there is a subject named Optimization Method that I found really interesting to me

\subsection{Is there any kind of technology you can use in study?}
if programming and math are kinds of technology

\section{Hometown}

\subsection{Has your hometown changed much these years?}
hugely; a better word;

the word change applies for most cities of China, after the reform and open up

\subsection{Is that a big city or a small place?}
it's a middle city;
can't compare to captial city?(mind grammar);

middle but with high happiness(mind grammar)

\subsection{How long have you been living here?}
I guess it's referring to my hometown;

18 years, before I enrolled in collage;

\subsection{For you, what benefits are there living in a big city?}
several advantages;

first, there are some kinds of jobs that do not exist in small city, which means more opportunities;

second, better infrustructure

and, universities

\subsection{Is there anything you dislike about it?}
last bus is at about 7 pm, a little inconvenient

\subsection{Where in your conutry do you live?}
Shanghai, cause I'm studying here

\section{Accommodation}

\section{Snacks}

\section{Birthday}

\section{Social Media}

\section{Names}

\section{Weather}

\section{Singing}

\section{Public Transport}

\section{Housekeeping and cooking}

\section{Geography}

\section{Puzzles}

\section{Writing}

\section{Morning routine}

\section{Technology}

\section{Feel Bored}

\section{Meeting Places}

\section{Computers}

\end{document}
