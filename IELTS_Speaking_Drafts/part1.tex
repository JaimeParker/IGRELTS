\documentclass[conference]{IEEEtran}
\IEEEoverridecommandlockouts
% The preceding line is only needed to identify funding in the first footnote. If that is unneeded, please comment it out.
\usepackage{cite}
\usepackage{amsmath,amssymb,amsfonts}
\usepackage{algorithmic}
\usepackage{graphicx}
\usepackage{textcomp}
\usepackage{xcolor}

% 添加中文支持包
%\usepackage[UTF8]{ctex}
%\usepackage{CJKutf8}

\def\BibTeX{{\rm B\kern-.05em{\sc i\kern-.025em b}\kern-.08em
    T\kern-.1667em\lower.7ex\hbox{E}\kern-.125emX}}


\begin{document}

\title{Speaking Part 1\\
{\footnotesize \textsuperscript{*}Note:
9-12 season query}
}

\author{\IEEEauthorblockN{Liu Zhaohong}
}

\maketitle

\begin{abstract}
speaking draft for IELTS test
\end{abstract}

\begin{IEEEkeywords}
IELTS, Speaking
\end{IEEEkeywords}


\section{Study/Work}

\subsection{Do you work or are you a student?}
Student + Degree + School + Major(engineering)

I'm a post graduate student at SJTU currently.

Majoring in aircraft control and information engineering. 


\subsection{What subject are you studying?}
Aircraft control and information engineering.

mostly it's about control stuff, like control law's designing and modelling of an aircraft

\subsection{Why did you choose that subject?}
I liked it in my high School and I was interested in planes, aircraft, something stuff

So, I chose this subject as my bachelor's Major and kept learning it in my master's study



\subsection{What would you like to do in the future?}
to be a robotics engineer



\subsection{What are the most popular subjects in China?}
computer science

seems with higher salaries than other subjects you can remember

\subsection{Do you think it's important to choose a subject you like?}
after the bachelor's time span, I'd say it depends on what life stage we are in

during bachelor time, one can choose subjects that are more universal.

For engineering students, they are math, physics, electronics and computer science etc.
the base for further study.

While in post graduate period, interest is rather important because
one need the interest to support his determination on doing research

\subsection{Are you looking forward to working?}
depends on the ??? of master degree

suitable for doing research or better for industry

\subsection{Do you like your subject? Why or Why not?}
no; compulsory subjects are not reasonable for us 

control vs. material mechanics

\subsection{Do you prefer to study in the mornings or in the afternoons?}
mornings

eat a lot at lunch, and digesting need more oxygen so I always feel sleepy

\subsection{Is your subject interest to you?}
mostly not, but there is a subject named Optimization Method that I found really interesting to me

\subsection{Is there any kind of technology you can use in study?}
if programming and math are kinds of technology

\section{Hometown}

\subsection{Has your hometown changed much these years?}
hugely; a better word;

the word change applies for most cities of China, after the reform and open up

\subsection{Is that a big city or a small place?}
it's a middle city;
can't compare to capital city?(mind grammar);

middle but with high happiness(mind grammar)

\subsection{How long have you been living here?}
I guess it's referring to my hometown;

18 years, before I enrolled in collage;

\subsection{For you, what benefits are there living in a big city?}
several advantages;

first, there are some kinds of jobs that do not exist in small city, which means more opportunities;

second, better infrastructure

and, universities

\subsection{Is there anything you dislike about it?}
higher living cost

and???

\subsection{Where in your country do you live?}
Shanghai, cause I'm studying here

but my hometown is in northern China

\section{Accommodation}

\subsection{Are the transport facilities in your city very good?}
not So

last bus is at about 7 pm, a little inconvenient;
environment on the bus is good

\subsection{Which room does your family spend most of the time in?}
sadly I guess it's the dining room, when we eat meals

used to stay at the living room, for news and weather forecast

due to smartphones, more likely to lie on each bedrooms

\subsection{Do you live in a house or a flat?}
flat, actually dorm(related to student question, where live and what status
I'm in)

\subsection{Do you plan to live here for a long time?}
of course not, house price is an astronomical number for meals

it's a considerable problem to choose a big city or small city,
and house price is definitely a shortcoming for big city

\subsection{Do you live alone or with your family?}
live alone(related to student question) in dorm
cause I'm still a student and it's not affordable for me to rent a room here

\subsection{How long have you lived there?}
there?here?

here: for only 3 months, just enrolled inconvenient

there: ?

\subsection{What do you usually do in your home/house/flat?}
\begin{itemize}
    \item home; I'm still a enrolled in student so 
    there isn't much time for me to stay at home;
    \item flat; the workload is stressful for me;
    and while staying at dorm, I tend to relax instead of getting to
    those unfinished works; so I would go to sw else to study
\end{itemize}

\subsection{Which is your favorite room in your home?}
living room, with a big scale television and a spacious sofa

feel good lying there watching my favorite episodes

\subsection{What's the difference between 
where you are living now and where you lived in the past?}
If the question before is about ever moved house:

facilities are all renewed and with larger areas

If the question before is about place living now and home:

dorm is totally different from home, in almost every aspect;
like no television and sofa for me to relax

\subsection{What can you see when you look out the window of your room?}
    \begin{itemize}
        \item home: in China, If you are not the kind of rich guy, then
        it's probably that the building you are living in is next to another
        building. Then there goes 2 consequences, facing the street, located
        at the edge of the community or facing another building in your community
        \item flat(dorm): it's quite crowded in campus, the scenery I can see
        from my flat's window is another flat building
    \end{itemize}

\subsection{Would you be willing to live in the countryside in the future?}
If the future is I'm rich enough and kind of old, I will;

fighting for my whole life seems a unpractical goal for me,
and I want to relax when I'm old and got no worry about living

\section{Snacks}
Note: New Questions for Season 9-12

\subsection{What snacks did you eat when you were young?}
A famous Chinese traditional food named LaTiao;

TODO: how to describe

\subsection{Do you often eat snacks now?}
no, snacks seems not so healthy;

I'd rather spend that money on other things like saving money
for a trip

\subsection{Do you think eating snacks is healthy?}
It depends actually;
% 在这里我想说,有时候一丁点零食能够带来很大的幸福感,这时候就不用担心健康与否
for me, I can get 

\section{Birthday}
\subsection{What do you usually do on your birthday?}
interesting, my birthday is near the date of returning to school

so prepare to get back to school or on the way to or just reach school

\subsection{What did you do on your birthday when you are young?}

\subsection{Do you think it is important for you to celebrate your birthday?}

\section{Social Media}
Note: New Questions for season 9-12
\subsection{Do you like social media?}
\subsection{Do you think your friends spend too much time on social media?}
\subsection{Do you want to work in a social media company?}

\section{Names}
\subsection{Does your name have a special meaning?}
\subsection{How do people choose names for their children?}
\subsection{Does anyone in your family have the same name with you?}
\subsection{Are there any differences between how Chinese name their
children now and in the past?}
%注意与part2的关联

\section{Weather}

\section{Singing}

\section{Public Transport}

\section{Housekeeping and cooking}

\section{Geography}

\section{Puzzles}

\section{Writing}

\section{Morning routine}

\section{Technology}

\section{Feel Bored}

\section{Meeting Places}

\section{Computers}

\end{document}
