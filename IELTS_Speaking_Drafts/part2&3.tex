\documentclass[conference]{IEEEtran}
\IEEEoverridecommandlockouts
% The preceding line is only needed to identify funding in the first footnote. If that is unneeded, please comment it out.
\usepackage{cite}
\usepackage{amsmath,amssymb,amsfonts}
\usepackage{algorithmic}
\usepackage{graphicx}
\usepackage{textcomp}
\usepackage{xcolor}
\usepackage{soul}

\begin{document}

\title{IELTS Speaking Part 2\&3\\
{\footnotesize \textsuperscript{*}Note: 9-12 season query}}

\author{\IEEEauthorblockN{Liu Zhaohong}}

\maketitle

\begin{abstract}
speaking draft for IELTS test part 2 and 3
\end{abstract}

\section{People}
\subsection{Describe a person you know who is from a different culture}
Leo Messi;
    \begin{itemize}
        \item Who he/she is
        \item Where he/she is from
        \item How you knew him/her
        \item How you feel about this person
    \end{itemize}

For me that person should be Leo Messi. I think almost everybody knows him or ever heard about him.
He is a football player from Argentina, with the nationality of both Spain and Argentina.
I heard stories about him in my primary school, probably the year 2010.
At that time Messi was still young, but he was already famous in the football area;

Firstly I thought Messi was a selfish man who only dribble the ball by himself and never passes it to his teammates,
and know nothing about the teamwork. But now, actually since 2014, I have become a big fun of him.
2014 is the year when the Brazil world cup were held.
If you are familiar with that cup, you must remember that the final was between Germany and Argentina,
and unfortunately, Argentina lost that game with Mario Gotze scored at 114 minute.
It's really sad because Messi and his teammates had several good chances before that goal.
However, they didn't make it, I mean the chance didn't convert to a score.
Sadly that match was just a start, Argentina lost another American cup final in the last 2 years.
For Messi himself, Barcelona was turned over twice at the European Champions League.

I remember once during that time, Messi quitted the nation team because suffering from so many failures.
But finally he came back again and led Argentina team for the 2018 World cup, sadly lost again.

Looking at Messi's career, you will see with such talents and diligence of him, failure still could
be a common thing. I like his dribbling skills, which is really attractive, and I also like his humble personality.
But most of all, I like his behavior that a man can be defeated, but sooner or later he should stand up
then facing and overcoming those obstacles.
This give me the bravery to handle my daily failure, too.

\textit{1. How can we get to know people from different cultures better?}

first of all, respect; which means keep a humble profile and better know some of his or her culture.
Especially the taboo in that culture;

Second one is to express any idea directly; That is to be simple, meet anything confusing, be free to say that
I don't know then ask;

For more or detailed understanding, read books or watch news. Movies is also a good way.

\textit{2. What are the advantages and disadvantages of cultural diversity?}

Advantages include that, one can know different kinds of culture and that could be interesting;
different culture brings various custom, life style, these are the treasure of all human being.

Disadvantage might be, more possible conflict.
But that is not for certain. Different culture might take misunderstanding, and even account for fight and blood.
However, that can be avoided by peaceful communication.   

\textit{3. How can traditional culture and other cultures coexist?}

by finding similarity and ignore difference as much as possible.
focus on the similarity or the double-win staff.


\subsection{Describe a person who is fashionable}
My girlfriend;

\begin{itemize}
    \item Who he/she is
    \item What he/she does
    \item What kind of clothes he/she wears
    \item And explain why you think this person is fashionable
\end{itemize}

Okay I'm gonna talk about my girlfriend, the closest fashionable person to me.
currently she is a postgraduate student like me, and we were schoolmates during the undergraduate period.
There is a slang in China that a swift horse needs a high quality saddle to utilize its strength, so does a man, 
who needs suitable clothes.

I think a woman, or paticularly my girlfriend, has 2 accounts;
one account is used in campus, another one is for entertainment, and they are quite different.
In campus she usually wears simple and comfortable clother, like sports clothes.
I mean a pair of sneakers, a pair of loose pants, some daily T-shirt or hoodie.
However, that's another story for the recreation part;
That include dress, all kinds of coats, jackets, shoes and so on.
By the way, also including different colors of dyed hair;

While the most fashionable thing about her is not the clothes that she wear, but the combination of them.
I mean, maybe each clothes is a simple one, without complex design or various logos and icons, some of
them even seems plain and simple.
But my girlfriend got the magic to organize them (mixture and fusion and integration).
It's like she is solving a math problem that by which means of arranging the combination will bring highest fashion score.

So fashion is kind of knowledge and witness I guess.
Unfortunately I'm the one who know nothing about fashion, so I'm lucky to have her as my coach on dressing.

\textit{1. Are older people as fashionable as young people?}

I think there is no coorelation between age and fashion.
My opinion is they may have diferent definition about what fashion is.
For example, people born in 1950s used to wear the same green uniform as the military,
to express their respect to PLA and that's their fashion.
China keeps developing, after the reform and opening up we accecpted more western cultures;
Till now, we define our own fashion, maybe each one have their own fashion.

\textit{2. Are women more fashionable than men?}

It seems so but I think this is a problem that does't have that obvious link with gender.

We may all believe that women spend more time on make up, to preen themselves.
But actually some men do it either.
It's not about gender about only related to whether this persion is a fashionable one or not.

\textit{3. Why is fashion more important to some people?}

I think being fashion make them feel more confident, and being confident means one can express him or herself
more bravely.

For example, being fashion for some introvert people is kind of mental hint, telling them they are good and be
free to say anything or do something, I guess.


\subsection{Describe a person you know who loves to grow plants}
some source said plants=vegetables and fruits, some add flowers

My dad;

\begin{itemize}
    \item Who this person is
    \item What he/she grows
    \item Where he/she grows those vegetable/fruits
    \item And explain how you feel about this person
\end{itemize}

That person is my dad. He was born in 1970s, and grew up in a small village in northern China.
His parents, namely my grandmother and father were farmers.

So he did a lot of farming works when he was young, and he kept studying during that time, 
finally with the reform and opening up and his hard work, he entered a agricultural university, majoring in vest.
But he didn't end up to be a vest, he was distributed to an anti-virus department after graduation.
Now he is working in a normal government department, nothing related to planting at all.

However, during these years, he still keeps his planting skills.
We lived in the top floor of a flat in the past. And we had an outdoor balcony there, 
so my father used that place as a natural field for planting Chinese chives(vegetables) and grapes.
and I remember each summer when I was young, we can eat our own grapes, although so many of them were eaten by
passing-by birds and bees, so actually only a few left for us.
We once kept a dog, my dad found him near a dustbin homelessly so he took him back home and raised him.
and that dog destroyed many of his vegetables, dad didn't blame him, we just didn't have our own chives to make
dumplings those days.

Except for the balcony stuff, dad also have many flowers inside the room, each time they blossom dad will take
pictures and share that to his friends.

I think the purpose of planting has changed for him.
In the past when he was young, he had to do farm work to help his family.
When he grew older and got to work in the city, he had no need for that farm work,
till now, watering the flowers and other works look like a hobby, or an addiction for him.
I guess that's because there is farmer's blood in his vein.

\textit{1. What do you think of the job of being a farmer?}

I think it's quite normal, being a farmer is also a job in the whole social life.
And I believe that there shouldn't be discrimination against farmers, ranking them as the lowest of this society;
reason is simple cause no farmers, no food, no anything.

\textit{2. Are there many people growing their own vegetables now?}

Farmers still occupy a considerable part of population in China, and they surely grow their vegetables.
But for people living in cities, I don't think many of them have the time, skill and space to grow their own vegetables.


\subsection{Describe your favorite childhood friend}
My neighbor;

\begin{itemize}
    \item Who he/she is
    \item Where you met each other
    \item What you often did together
    \item And explain what made you like him/her
\end{itemize}
% who and where met
The childhood friend that I'm gonna talk about is my neighbor's son. He is also my neighbor either.
You know this neighbor I'm saying is not we lived in the same building but, in the same floor and door to door.
He is 4 years older than me, and I don't remember when we met.
Because I guess when we met is really a historical problem, maybe his mother took him to see my mother and me
just after I was born safely.

% often did together
We did a lot of things together, but there are some remarkable ones.
When I was 7 or 8, in my primary school period. Some buildings were dismantled near my home, because
they are too old and the city wanted to build some new stuff.
You know, the dismantle and construct departments are 2 different department.
So there is quite a time span that the ruin stayed there. And one night, we picked up some branches
, papers and a lighter to set up fire for barbecue.
and the food we baked is a corn.
I remember the lighting process is quite hard cause it's a little windy so we managed to light the paper first;
then using the burning paper to light branches, finally using steady fire to bake corn.
Actually that corn wasn't that delicious, but the process is interesting for me.
% digress a little
We also used to make paper plane then throw them at my outdoor balcony, to see whose plane could fly further.
and sometimes rode bicycle around the city in some nights. 

% what made like
To be honest, set up a fire on a mess of ruins is really dangerous. And except for that frenzy stuff,
he is also kind of my path-leader, after all he is older than me, so he did teach me some experience
that I used later.

\textit{1. Why do people lose contact with their friends after graduation?}

It has a lot of reasons;
First, we may spend more time with people around us. so without particular attention,
we might just forget to communication with old friends.

Secondly, if people go to work after graduation, the work load may cut greatly on their time
for social, causing the losing contact of old friends. 


\textit{2. How does modern technology influence friendship?}

for truly friendship, technology makes 2 people get to know each other more quickly and precisely.
It's beneficial for keeping the friendship.

In the meantime, the time or cost to interrupt someone is also declined. If you make a call to your
friend directly while he is having an important conference with his volume on, that could be a miserable.

\textit{3. Do you think people's relationship with friends will change when they get older?}

I think this isn't about age, or have little to do with age. In my opinion, there are several factors deciding the friendship.

\begin{itemize}
    \item life path;
    \item values towards life and world;
    \item political stand;
    \item personality humble and arrogant.
\end{itemize}

\subsection{Describe a person who inspired you to do something interesting}
Leo Messi

\begin{itemize}
    \item Who he/she is
    \item How you knew him/her
    \item What interesting thing you did
    \item And explain how he/she inspired you to do something interesting
\end{itemize}

% who 
For me that person should be Leo Messi. I think almost everybody knows him or ever heard about him.
He is a football player from Argentina, with the nationality of both Spain and Argentina.
I heard stories about him in my primary school, probably the year 2010.
At that time Messi was still young, but he was already famous in the football area;

% how know
Firstly I thought Messi was a selfish man who only dribble the ball by himself and never passes it to his teammates,
and know nothing about the teamwork. But now, actually since 2014, I have become a big fun of him.
2014 is the year when the Brazil world cup were held.
If you are familiar with that cup, you must remember that the final was between Germany and Argentina,
and unfortunately, Argentina lost that game with Mario Gotze scored at 114 minute.
It's really sad because Messi and his teammates had several good chances before that goal.
However, they didn't make it, I mean the chance didn't convert to a score.

% what thing did
Admiring his skill of dribbling and shooting, I started play football after the Brazil world cup.
It's Messi led me the football area, get to know this sport.
I wasn't a technical player, the matches I involved ended up with loss and wins, so did some matches
that you can see on the TV.
I think the most attractive and interesting part of football is its uncertainty and possibility.
Like this time, Argentina lost to Saudi Arabic in the first group stage.
Like years ago, in the last 5 minutes, Barcelona scored 3 goals and defeated Paris Saint German.
That's the way he led me in and I found interesting.

\textit{1. What qualities make someone a role model?}

\begin{itemize}
    \item defeated but stand up again; 
    \item diligent; 
    \item chances to success; always fail is respectful, but the unknown let them lose the chance to be a model.
\end{itemize}

\textit{2. Who can influence children more, parents or teachers?}
% TODO
\begin{itemize}
    \item 
\end{itemize}

\textit{3. Why should children learn from role models?}

\begin{itemize}
    \item like solving a math problem, first step is to learn the example, then extrapolating to other aspects;
    \item role models have some good quality;
    \item admire is a way to learn;
\end{itemize}


\section{Object}
% https://zhuanlan.zhihu.com/p/110583340
\subsection{Describe a photo you took and you are proud of}

\begin{itemize}
    \item \textbf{When you took it}
    
    When I was a senior student in my undergraduate period, 
    2021's winter, 
    third year in my undergraduate life.

    \item \textbf{Where you took it}
    
    An old temple in Xi'an city, the foot of Qinling Mountains.

    \item \textbf{What is in the photo}
    \item \textbf{Explain why you are proud of it}
\end{itemize}



\subsection{Describe an object that you think is beautiful}
\subsection{Describe a movie you watched recently and would like to watch again}
\subsection{Describe a program you like to watch}
\subsection{Describe an important thing you learned(not at school or university)}
\subsection{Describe a story or a novel that you have read and you found interesting}


\section{Place}
\subsection{Describe a popular place for doing sports}
\subsection{Describe the home of someone you know well and you often visit}
\subsection{Describe a place in your country that you would like to recommend to
travelers}

\section{Event}
\subsection{Describe a problem you had while shopping online or in a store}
\subsection{Describe a time when you made a decision to wait for something}
\subsection{Describe a time when you received money as a gift}
\subsection{Describe a disagreement you had with someone}
\subsection{Describe an outdoor activity you did in a new place recently}
\subsection{Describe a time when you forgot an appointment}
\subsection{Describe a time when you shared something with others}
\subsection{Describe a time when you needed to search for some information}
\subsection{Describe a time when you saw a lot of plastic waste (e.g. in a park, on the beach
etc.)}
\subsection{Describe a time when you enjoyed an impressive English lesson}
\subsection{Describe a difficult thing you did and succeeded}




\end{document}
